\documentclass[twoside]{article}
\setlength{\oddsidemargin}{0in}
\setlength{\evensidemargin}{0in}
\setlength{\textwidth}{13.0cm}
\setlength{\textheight}{19.5cm}
\pagestyle{myheadings}
\begin{document}
\thispagestyle{headings}
\begin{center}
%Title:
\Large \bf Submitting  a Poster Abstract to RECOMB 2003\\
%authors, affiliations and addresses

\vspace{.5cm}\normalsize 

Kathryn Janeway,
\footnote{Institute of Quantum Cosmology, Starfleet Academy, Mark Road, Indiana, 
North America, Earth. E-mail: {\tt janeway@qc.starfleet.ea}} 
Jean-Luc Picard,
\footnote{School of Archaeology, Starfleet Academy, Rue William Shakespeare 1, 
LaBarre, France, Earth. E-mail:  {\tt picard@ar.starfleet.ea}}
James T. Kirk
\footnote{Dept. of Inspections and Training Operations, Starfleet Academy,  
Spock Street 1, Klingonia, Nexus. E-mail:
{\tt kirk@it.starfleet.ne}} 
\end{center}

\small

\vspace{.3cm}

\noindent{\bf Keywords:} federation, stardate, class-M-environment, phaser, vulcano,
 beam me up, scottie

\section{\large Introduction.} Poster abstracts should be submitted 
electronically. The abstracts of accepted posters will be published in book
form and will be available at the meeting. The following instructions are
intended to maximize space available to authors while ensuring agreeable and
consistent appearance, efficient handling and timely publication. Final posters
may, of course, be in any format, as long as they fit onto a board of 1.32 m 
height and 0.90 m width (52.0 x 35.4 inches). Acceptance of an abstract is conditional on at 
least 
one of its authors preregistering for RECOMB electronically before February 3, 2003 (see
{\tt http://www.recomb2003.de}).

Abstracts must be received by January 6, 2003; authors will be notified about
acceptance before January 27, 2003. In the case of files we cannot open or
which do not satisfy the format instructions, we will try to notify the authors
within a few days of receipt to allow for resubmission.  Authors who cannot
comply with software or e-mail instructions should notify
{\tt beziat@molgen.mpg.de} well in advance of the deadline.

\section{\large Software and files.}The abstract, written in clear and
reasonably grammatical English, must be submitted in exactly the same format as
the example in the templates, and should fit on two pages, everything included. 
If at all possible, use LaTeX 2e article document class, with all the default
options, and with a minimum of other packages and other files.  Download the
template file provided. To conserve space, do not split the abstract into many
small sections. 

For those for whom LaTeX is impossible, use Word or other easily translatable
word-processing software.  Download the Word template and substitute your own
text, titles and headings, conserving the same fonts, sizes and styles.

Submit the LaTeX source file (or Word file), AND either a postscript version or
a .pdf version, or both.  These should be sent to {\tt
beziat@molgen.mpg.de} as attachments.  The subject header of each message
should identify the paper by the first author's surname and the type of file it
contains.  Authors submitting more than one abstract should disambiguate them
using numerals.  E.g. smith1.tex, smith1.ps, smith2.tex, smith2fig1.eps,
smith2.pdf, smith2.ps, smith3.doc.

\section{\large Figures and tables.} Use standard LaTeX figure and table
environments.  Make sure figures are readable and not too crowded. Do not use
type sizes smaller than {\bf footnotesize} in figures and tables.  No colour
please.  For LaTeX submissions, figures must be composed in LaTeX or
incorporated in the text using .eps files. 

Figures in Word submissions must be incorporated in the abstract, or sent as
.eps files.  In any case, they must appear properly positioned in  .ps and .pdf
files. 

Use the general format in Figure \ref{circles}, if possible. 
\begin{figure}[h]
\begin{center}
\footnotesize
\begin{picture}(300,60)
\put(30,20){\circle{40}}
\thicklines
\qbezier(36,40)(45,38)(50,26)
\put(120,20){\vector(1,0){50}}
\thinlines
\put(60,35){turn of length $f$}
\put(60,35){\vector(-2,-1){10}}
\put(20,45){\line(1,-2){5}}
\put(20,45){\line(1,-2){5}}
\put(28,45){$d$}
\put(40,33){\line(1,2){5}}
\put(240,20){\circle{40}}
\put(230,45){\line(1,-2){5}}
\put(255,25){\line(1,0){10}}
\put(248,41){$d'$}
\end{picture}
\end{center}
\caption{\footnotesize Effect of turns of length $f$ on the distance from $d$ to
$d'$}
\label{circles}
\end{figure} 

Tabular displays should have the general format of Table \ref{turns}, if
possible.  

\begin{table}[h]
\begin{center} {\footnotesize
\begin{tabular}{|c|cc|cc|cc|cc|}
\hline
 & \multicolumn{2}{c|}{$x=10.3$} & \multicolumn{2}{c|}{$x=10.5$} &
\multicolumn{2}{c|}{$x=21.7$} & \multicolumn{2}{c|}{$x=22.8$} \\
$q$  & \multicolumn{1}{c}{gmt} & \multicolumn{1}{c|}{jfk} &
\multicolumn{1}{c}{fbi} & \multicolumn{1}{c|}{mac} & \multicolumn{1}{c}{fax} &
\multicolumn{1}{c|}{era} & \multicolumn{1}{c}{ibm} &
\multicolumn{1}{c|}{pdf}\\\hline
$a/20$ &     117.01 & 117.02  &   318.31 & 318.27  &   29.83 & 29.85    & 36.35 
& 36.40 \\
$d/40$  &     338.88 & 338.03  &   53.34 & 53.31  &  97.29 & 67.24 &   26.74 &
126.52 \\
$5e$ &     246.58 & 246.50  &   47.57 & 47.57  &  24.28 & 24.89 &   55.54 &
155.26 \\
$7f$    &     157.64 & 157.54  &   36.83 & 36.83  &  38.80 & 38.02 &   72.63 &
172.60 \\
$2g/3$   &     58.08 & 58.05  &   27.35 & 27.32  &  56.20 & 56.23 &   95.47 &
195.49 \\
$\lceil \log h \rceil h^2$    &     18.99 & 18.99  &   99.00 & 98.99  &  58.94 &
158.99 &   98.99 & 98.99 \\
\hline m.t. &     4.6 &  5.1 &     3.8 &  4.1 &     2.2 & 1.9  &   3.6 &  3.7 \\
\hline
\end{tabular} }
\end{center}
\caption{\footnotesize Number of turns and distance between top and bottom.}
\label{turns}
\end{table}

The .ps and .pdf files must include all illustrations and tables as part of the
two-page limit.

\section{\large References and bibliography.}

Use the styles exemplified by \cite{HB98}, \cite{CA}, \cite{MSW00} and
\cite{Rei91} below.  Order bibliography items alphabetically as shown.  The
two-page limit includes the bibliography.
 
\footnotesize 
 \begin{thebibliography}{99}
 
\bibitem{HB98} Huynen, M.~A. and Bork, P. 1998. Measuring genome evolution. {\em
Proceedings of the National Academy of Sciences USA}
  95:5849--5856.

\bibitem{CA} Caprara, A. 1997. Sorting by reversals is difficult. In: {\em
Proceedings of the First Annual International Conference on Computational
Molecular Biology (RECOMB 97),} New York: ACM.  pp. 75-83.

\bibitem{MSW00}McLysaght, A., Seoighe, C. and Wolfe, K.~H. 2000. High frequency
of inversions during eukaryote gene order evolution.     In Sankoff, D. and
Nadeau, J.~H., editors, {\em Comparative Genomics},  Dordrecht, NL: Kluwer
Academic Press. pp. 47--58.

\bibitem{Rei91} Reinelt, G. 1991. {\em The Traveling Salesman - Computational
Solutions for TSP Applications.} Berlin: Springer Verlag. 
 
\end{thebibliography}

\end{document}
